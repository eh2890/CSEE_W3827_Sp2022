\documentclass{../slides}

\title{3827 OH}
\author{Eumin Hong (eh2890)}
\institute{Columbia University}
\date{March 29, 2022}

\usepackage{mips}
% MIPS lstlistings
\lstdefinestyle{MIPS}{frame=tb,
  language=[mips]Assembler,
  basicstyle=\footnotesize,
  numbers=left,
  numberstyle=\tiny\color{gray},
  stepnumber=1,
  numbersep=5pt,
  backgroundcolor=\color{white},
  showspaces=false,
  showstringspaces=false,
  showtabs=false,
  rulecolor=\color{black},
  tabsize=4,
  captionpos=b,
  breaklines=true,
  breakatwhitespace=false,
  title=\lstname,
  keywordstyle=\color{blue},
  commentstyle=\color{dkgreen},
  stringstyle=\color{mauve},
  escapeinside={\%*}{*)},
  morekeywords={*,...}
}

\lstset{style=mips}

\begin{document}

\begin{frame}
    \titlepage
\end{frame}

\begin{frame}{Overview}
\begin{multicols}{2}
\tableofcontents
\end{multicols}
\end{frame}

\section{Announcements}
\subsection{Upcoming Exams and Homework}
\begin{frame}{\secname: \subsecname}
    \begin{itemize}
        \item HW5 due 4/1 (Ed post \#252)
        \item HW6 due 4/4 (Ed post \#256)
    \end{itemize}
\end{frame}

\subsection{Midterm}
\begin{frame}{\secname: \subsecname}
    \begin{itemize}
        \item Do not throw away your exam if you plan to request a regrade
        \item Form: \url{https://forms.gle/8S8WYGD2B8LSsRmt9}
        \item I cannot comment on how a certain question was scored since I may not have scored it
        \item I do not know how the different exams are relatively curved
    \end{itemize}
\end{frame}

\subsection{Feedback}
\begin{frame}{\secname: \subsecname}
    \begin{itemize}
        \item Form: \url{https://forms.gle/cnUmKVNYN7WvRbHA6}
    \end{itemize}
\end{frame}

\section{Homework 5 Material}
\subsection{Overview and Relevant Lectures}
\begin{frame}{\secname: \subsecname}
    \begin{itemize}
        \item All Part $n$ references are for Part $2.n$ in the \enquote{Coding Details} part of HW5 (i.e. the programming parts)
        \item Introduction to MIPS programming (Lecture 07, Slides 35-81)
        \item Conditional logic in MIPS (Lecture 07, Slides 82-95)
        \item Stack pointer and recursion (Lecture 07, Slides 96-111)
        \item MIPS calling conventions (Lecture 07, Slides 112-166)
        \item For Part 3 (\lstinline{AddAndVerify}), use new testing program \lstinline{test-AddAndVerify-plus.s} on CourseWorks
        \begin{itemize}
            \item The original does not print any decrypted string even if you are correct
            \item Also original says \enquote{ALL DONE} even if you are wrong
            \item It is fairly clear when the new testing code outputs correct string
        \end{itemize}
        \item For the main message (Part 4), the last four characters can be ignored
    \end{itemize}
\end{frame}

\subsection{General Part 3 Structure}
\begin{frame}[fragile]{\secname: \subsecname}
    \begin{lstlisting}
    AddAndVerify:
        # Base case checking/branching
        # If not base case, push to stack using `sw`
        # Other necessary operations
        jal AddAndVerify
        # Pop from stack
        # Check $v0 to see if suffix is valid for branching
        # Other necessary operations
        jal WordDecrypt
        # Other necessary operations
        jal IsCandidate
        # Other necessary operations such as writing to destination address
        jr $ra
    \end{lstlisting}
    \begin{itemize}
        \item This is just a rough outline -- you must fill in the rest of the code
    \end{itemize}
\end{frame}

\section{Homework 6 Material}
\subsection{Relevant Lectures}
\begin{itemize}
    \item Coincident selection (Lecture 10, Slides 41-58)
    \item Scaling memory using multiple chips (Lecture 10, Slides 59-85)
\end{itemize}

\end{document}
