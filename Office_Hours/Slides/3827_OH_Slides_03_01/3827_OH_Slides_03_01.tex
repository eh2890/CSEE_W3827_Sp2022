\documentclass{../slides}

\title{3827 OH}
\author{Eumin Hong (eh2890)}
\institute{Columbia University}
\date{March 1, 2022}

\begin{document}

\begin{frame}
    \titlepage
\end{frame}

\begin{frame}{Overview}
\begin{multicols}{2}
\tableofcontents
\end{multicols}
\end{frame}

\section{Announcements}
\subsection{Upcoming Exams and Homework}
\begin{frame}{\secname: \subsecname}
    \begin{itemize}
        \item Midterm is March 10 or 11 (depends on section/form you filled out at the beginning of the semester)
        \item HW3 graded, will have HW4 graded ideally over the weekend (after all submissions received)
    \end{itemize}
\end{frame}

\subsection{Homework 3 Feedback}
\begin{frame}{\secname: \subsecname}
    \begin{itemize}
        \item Common errors from Homework 3:
        \begin{itemize}
            \item Be sure to label the inputs/outputs of your decoders, MUXes, and similar circuitry (better safe than sorry)
            \item Use \enquote{don't cares} in Q3 -- whenever you have various possible inputs combinations for the same output, think about using \enquote{don't cares} (i.e. $10$ and $11$ both result in red lamp color)
            \item For questions like Q4/Q5, try creating some additional example strings
            \begin{itemize}
                \item Exam will not provide examples of all edge cases (but the behavior will be well-defined)
            \end{itemize}
        \end{itemize}
    \end{itemize}
\end{frame}

\subsection{Feedback}
\begin{frame}{\secname: \subsecname}
    \begin{itemize}
        \item Form: \url{https://forms.gle/cnUmKVNYN7WvRbHA6}
    \end{itemize}
\end{frame}

\section{Midterm Review}
\subsection{Overview}
\begin{frame}{\secname: \subsecname}
    \begin{itemize}
        \item The following slides list out the topics and techniques for notable homework questions and all exam questions
        \begin{itemize}
            \item Not going to talk about Warmup Problems -- those are pretty straightforward and covered in other problems
        \end{itemize}
        \item You should know how to solve all of the HW problems and exam problems
        \item All references to lectures and slides are based on the GitHub repository (updated as of March 1)
    \end{itemize}
\end{frame}

\subsection{Homework 1}
\begin{frame}{\secname: \subsecname}
    \begin{itemize}
        \item All conversions to and from binary should be second nature
        \item Q3: should know how to correctly detect overflow for both binary and 2's-complement addition algorithms (Lecture 01, Slides 67-68)
        \begin{itemize}
            \item % TODO: add relevant exam question about 1's complement
        \end{itemize}
        \item Q5: should know how to compare numbers (in all forms of binary, but the structure of floating point numbers makes it easy to compare) (Lecture 01, Slide 79)
    \end{itemize}
\end{frame}

\subsection{Homework 2}
\begin{frame}{\secname: \subsecname}
    \begin{itemize}
        \item Q1/Q2: should be able to simplify to minimize literals (Lecture 02, Slides 16-30)
        \begin{itemize}
            \item Should also be second nature (except detecting XORs/XNORs, which take more time)
            \item If stuck, try using K-maps % TODO: insert lecture ref
        \end{itemize}
        \item Q3/Q4/Q5: De Morgan's Law (Lecture 02, Slides 22-29)
        \item Q6/Q7/Q8: Boolean expressions/minterms to K-maps to Boolean expressions (Lecture 03, Slides 32-67)
        \begin{itemize}
            \item \enquote{Don't cares} (Lecture 03, Slides 71-81)
            \item Filling out a K-map should take no longer than $30$ seconds (rough estimate)
            \item Practice -- use friends or online resources to check your work
            \begin{itemize}
                \item Online K-map resource: \url{http://www.32x8.com/index.html}
            \end{itemize}
        \end{itemize}
    \end{itemize}
\end{frame}

\subsection{Homework 3}
\begin{frame}{\secname: \subsecname}
    \begin{itemize}
        \item Q1: using smaller combinational circuitry to build larger combinational circuitry (Lecture 04, Slides 52-97)
        \item Q2: meaning of minterms (Lecture 03, Slides 16-18)
        \item Q3/Q4/Q5: word problem to truth table/K-map to Boolean expressions (Lecture 04, Slides 98-104)
        \begin{itemize}
            \item Verify that you know what your inputs and outputs are (e.g. inputs are $ABCD$, outputs are $NSH, NSL, EWH, EWL$
        \end{itemize}
    \end{itemize}
\end{frame}

\subsection{Homework 4}
\begin{frame}{\secname: \subsecname}
    \begin{itemize}
        \item Q2/Q3: word problem to truth table/K-map to Boolean expressions for FSMs (Lecture 06, Slides 24-42 and 43-71)
    \end{itemize}
\end{frame}

\subsection{Spring 18 Midterm}
\begin{frame}{\secname: \subsecname}
    \begin{itemize}
        \item Q1: word problem to truth table/K-map to Boolean expressions (Lecture 04, Slides 98-104)
        \begin{itemize}
            \item If input is \enquote{unused}, think \enquote{don't care}
        \end{itemize}
        \item Q2: unsigned binary comparison
        \begin{itemize}
            \item Q2b: using pre-defined circuits to implement more complex circuit (means that this is independent of implementation in Q2a for base case)
            \item Q2c: MUXes to handle casework
            \item Q2d: 2's complement binary comparison
            \item Q2e: similar to ripple carry adder (Lecture 04, Slides 115-117)
        \end{itemize}
        \item Q3: not relevant
    \end{itemize}
\end{frame}

\subsection{Fall 18 Midterm}
\begin{frame}{\secname: \subsecname}
    \begin{itemize}
        \item Q1:
        \begin{itemize}
            \item Q1a: binary addition (Lecture 01, Slides 10-21)
            \item Q1b: MUXes to handle casework
        \end{itemize}
        \item Q2: 1's and 2's complement
        \begin{itemize}
            \item Q2a: conversion from binary to 1's complement and binary to 2's complement (Lecture 01, Slides 37-46)
            \item Q2b: $k$-bit addition and reusing inputs (Lecture 04, Slides 105-114)
            \item Q2c: MUXes to handle casework
        \end{itemize}
        \item Q3: word problem to truth table/K-map to Boolean expressions (Lecture 04, Slides 98-104)
    \end{itemize}
\end{frame}

\subsection{Spring 19 Midterm}
\begin{frame}{\secname: \subsecname}
    \begin{itemize}
        \item Q1:
        \begin{itemize}
            \item Q1a: unsigned binary overflow (for addition, not subtraction: Lecture 01, Slides 29-36)
            \item Q1b: detecting overflow (Lecture 01, Slides 67-68)
            \item Q1c: unsigned binary subtraction overflow
            \item Q1d: MUXes to handle casework
            \item Q1e: \enquote{A Slick MUX trick} (Lecture 04, Slides 78-82)
        \end{itemize}
        \item Q2: identifying behavior of latch given inputs (Lecture 05, Slides 12-17, 18-24, and 25-27)
        \item Q3: word problem to truth table/K-map to Boolean expressions for FSMs (Lecture 06, Slides 24-42 and 43-71)
        \begin{itemize}
            \item Q3a: determining FSM (Lecture 06, Slides 24-32 and 44-50)
            \item Q3b: filling in truth table for given flip-flop and getting Boolean expression (Lecture 06, Slides 33-35, 36-37, 38-42, and 56-71)
        \end{itemize}
    \end{itemize}
\end{frame}

\subsection{Spring 21 Midterm}
\begin{frame}{\secname: \subsecname}
    \begin{itemize}
        \item Q1: full adder (Lecture 04, Slides 105-114)
        \begin{itemize}
            \item Q1a: contraction (Lecture 04, Slides 127-134)
            \item Q1b: MUXes to handle casework
        \end{itemize}
        \item Q2:
        \begin{itemize}
            \item Q2a: conversion from binary to 1’s complement and binary to 2’s complement (Lecture 01, Slides 37-46)
            \item Q2b: Range of representation of binary numbers (Lecture 01, Slides 58) and 2's complement overflow (Lecture 01, Slide 68)
        \end{itemize}
        \item Q3:
        \begin{itemize}
            \item Q3a: determining FSM (Lecture 06, Slides 24-32 and 44-50)
            \item Q3b: filling in truth table for given flip-flop and getting Boolean expression (Lecture 06, Slides 33-35, 36-37, 38-42, and 56-71)
        \end{itemize}
    \end{itemize}
\end{frame}

\subsection{Other Important Topics and References}
\begin{frame}{\secname: \subsecname}
    \begin{itemize}
        \item Bitwise Operations (Lecture 02, Slides 39-42)
        \item XOR Gates (Lecture 02, Slides 9-10 and 54-55)
        \item NAND and NOR Gates (Lecture 02, Slides 43-45)
        \item SoP and PoS (Lecture 03, Slides 4-9)
        \item Minterms and Maxterms (Lecture 03, Slides 10-28)
        \item Implicant Terminology (Lecture 03, Slide 57)
        \item \enquote{Standard Circuits} (Lecture 04, Slides 35-51)
        \item Adder Circuits (Lecture 04, Slides 105-114 and 115-126)
        \item Contraction (Lecture 04, Slides 127-134)
        \item Latches (Lecture 05, Slides 4-52)
        \item Flip Flops (Lecture 05, Slides 53-83)
        \item Finite State Machines (Lecture 06, Slides 18-71)
    \end{itemize}
\end{frame}

\end{document}
